\chapter{Resultados Preliminares}

De maneira geral, a medida que aumenta-se o número de pares também aumenta a eficiência (Tabela~\ref{tab:p2p}). O \textit{rank} 0, que não realiza processamento, começa a representar uma porcentagem cada vez menor para o calculo da eficiência. Considerando 3 pares, o \textit{rank} 0 representa 33\% do calculo da eficiência, já considerando 5 pares o mesmo representa 20\%, assim a medida que aumentam-se os pares a eficiência aumenta. Está melhoria ocorre até certo ponto, pois a comunicação entre processos aumenta e medida que adicionamos novos pares. É importante ressaltar que esta particularidade ocorre devido ao \textit{rank} 0 não processar nenhuma atividade e participar do calculo de eficiência, e com o aumento do números de pares a eficiência diminua devido as taxas de comunicação. Se desconsiderado o \textit{rank} 0, a eficiência observada inicia elevada e na medida que aumentam-se os pares a eficiência deve diminuir devido à comunicação entre processos da rede.

\begin{table}[!h]
\caption{ Resumo dos resultados modo P2P - 10 \textit{Folds} }
\label{tab:p2p}
\centering
\setlength{\tabcolsep}{5pt}
\begin{tabular}{ccccccc}
\hline
Grupo &Grupo 1 &Grupo 2 &Grupo 3 &Grupo 4 &Grupo 5 &Grupo 6 \\
\hline
Experimento &Sequencial &3-Pares &5-Pares &7-Pares &9-Pares &11-Pares \\
\hline
Média &301,92 &146,07 &91,22 &61,44 &60,98 &31,97 \\
\hline
Desvio Padrão &1,05 &0,88 &1,10 &0,69 &0,75 &0,82 \\
\hline
\multicolumn{2}{c}{\textit{SpeedUp}} &2,07 &3,31 &4,91 &4,95 &9,44 \\
\hline
\multicolumn{2}{c}{Eficiência} &0,69 &0,66 &0,70 &0,55 &0,86 \\
\hline
\end{tabular}
\\
\singlespacing
\text{\footnotesize Fonte: O autor}
\end{table}

Utilizando as técnicas de mineração de dados, RNA e o método de validação cruzada, aplicados ao conjunto de dados de tipos de coberturas florestais foram obtidos os resultados presentes na tabela \ref{tb:resulcla}, expressos pelos índices de acerto, erro de classificação e índice Kappa.

\begin{table}[!h]
\caption{ Precisão do Classificador }
\label{tb:resulcla}
\centering
\setlength{\tabcolsep}{5pt}
\begin{tabular}{ccc}
\hline
Resultado &10 \textit{Folds} &24 \textit{Folds} \\
\hline
Índice de Acerto &15833 - 82.33\% &15921 - 82.79\% \\
\hline
Índice de Erro &3396 - 17.66\% &3308 - 17.20\% \\
\hline
Índice Kappa &0.794 &0.7993 \\
\hline
\end{tabular}
\\
\singlespacing
\text{\footnotesize Fonte: O autor}
\end{table}

Observando os percentuais de acerto na utilização das duas diferentes configuração de \textit{folds}, não se verifica uma grande diferença entre os resultados. Com 24 \textit{folds} o classificador acertou 88 instancias a mais que utilizando 10 \textit{folds}, que corresponde cerca de 0,45\% a mais de precisão.

\citeonline{Bla99}, utilizando os mesmo parâmetros da RNA, obtiveram uma precisão de 70,58\% com o modelo, porém não pode-se dizer que os resultados obtidos neste trabalho são melhores que os observador por Blackard e Dean, devido aos fatores: seleção de dados e método de validação empregado.

Além do percentual de acerto, também foi observado o coeficiente estatístico denominado índice Kappa ou Estatística K, definido como uma medida de concordância em escalas nominais. Neste contexto de classificação, verificando os altos valores do índice Kappa, podemos verificar o elevado nível de concordância entre a classificação do modelo e a classificação de referência, ou seja, o quão os dois estão de acordo quanto à classificação, novamente sendo observado um maior valor para 24 \textit{folds}.

Observando a área ROC, tabela \ref{tb:resuroc}, que retrata o desempenho de um classificador sem levar em conta os custos de distribuição ou de classe de erro, os resultados são muito expressivos, tendo em vista que todos os valores para área ROC estão superiores a 0.9.

\begin{table}[!h]
\caption{ Detalhes da Precisão por Classe }
\label{tb:resuroc}
\centering
\setlength{\tabcolsep}{5pt}
\begin{tabular}{ccccc}
\hline
Tipo de Cobertura &\multicolumn{2}{c}{Área Roc} &\multicolumn{2}{c}{Precisão}   \\
Florestal &10 \textit{Folds} &24 \textit{Folds} &10 \textit{Folds} &24 \textit{Folds} \\
\hline
Classe 1 &0.941 &0.942 &0.749 &0.761 \\
Classe 2 &0.924 &0.925 &0.687 &0.704 \\
Classe 3 &0.963 &0.965 &0.805 &0.798 \\
Classe 4 &0.995 &0.996 &0.927 &0.926 \\
Classe 5 &0.987 &0.985 &0.876 &0.861 \\
Classe 6 &0.973 &0.974 &0.776 &0.798 \\
Classe 7 &0.994 &0.994 &0.927 &0.929 \\
\hline
\textbf{Média} &\textbf{0.968} &\textbf{0.969} &\textbf{0.821} &\textbf{0.825} \\
\hline
\end{tabular}
\\
\singlespacing
\text{\footnotesize Fonte: O autor}
\end{table}
