\begin{abstract}

{\noindent

The objective of this study is investigate the use of parallel computing to reduce the response time of data mining in agriculture. For this purpose, a tool, called Fast Weka been defined and implemented. This tool allows running data mining algorithms and explore parallelism in multi-core computers with the use of threads and in computational grids employing peer-to-peer networks. The exploration of parallelism occurs through the data parallelism inherent to the process of cross-validation (folds). The tool was evaluated through experiments using artificial neural networks data mining algorithms applied to a data set of forest cover types. The multi-thread computing and computing on peer-to-peer networks allow to reduce the response time of data mining activities. The best results are achieved when employed a multiple number of threads or pairs in the number of folds of cross validation. It was observed an efficiency of 87\% when used 4 threads to 24 folds and 86\% efficiency also in peer-to-peer networks using 24 folds with 11 pairs.\\

}

{\noindent \textbf{Keywords:} Parallel Computing, Data Mining, Peer-to-Peer, Forest Cover Types}
\end{abstract}
