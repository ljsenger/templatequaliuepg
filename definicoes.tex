\usepackage{abnt-UEPG}
\usepackage[top=3cm,bottom=2cm,left=3cm,right=2cm]{geometry}
\usepackage[utf8]{inputenc}
\usepackage[brazil]{babel}
\usepackage[T1]{fontenc}
%\usepackage{hyperref}
\usepackage{amssymb}
\usepackage{ae}
\usepackage{amsmath}
\usepackage{paralist}
\usepackage{graphicx}
\usepackage{subfigure}
\usepackage{setspace}
\usepackage{fancyhdr}
\usepackage{lscape}
\usepackage{longtable}
\usepackage{fancyvrb}
\usepackage{gantt}
\usepackage[alf,recuo=0cm,abnt-etal-text=it,bibjustif, abnt-emphasize = bf]{abntcite}
\usepackage{listings}
\renewcommand{\lstlistingname}{Código} % definição visual dos códigos fonte
\usepackage{inconsolata}
\usepackage[T1]{fontenc}
\usepackage{color}
\usepackage{lipsum}% just to generate text


% % % % % % % % %
%\usepackage[explicit]{titlesec}
%\usepackage{xcolor}
%\usepackage{lipsum}% just to generate text

%\colorlet{myrulecolor}{black}
%\definecolor{myrulecolor}{RGB}{150,20,0}% define the color for the rules

%\titleformat{\chapter}[display]
%  {\normalfont\scshape\Huge}
%  {\hspace*{-70pt}\thechapter.~#1}
%  {-15pt}
%  %{\hspace*{-110pt}{\color{myrulecolor}\rule{\dimexpr\textwidth+80pt\relax}{3pt}}\Huge}
%\titleformat{name=\chapter,numberless}[display]
 % {\normalfont\scshape\Huge}
 % {\hspace*{-70pt}#1}
 % {-15pt}
 % %{\hspace*{-110pt}{\color{myrulecolor}\rule{\dimexpr\textwidth+80pt\relax}{3pt}}\Huge}
%\titlespacing*{\chapter}{0pt}{0pt}{30pt}

%\titleformat{\subsection}{\normalsize\bfseries\itshape}{\textnormal{\roman{subsection}.}}{1em}{}
%\titleformat{\section}{\normalsize\bfseries\itshape}{\textnormal{\roman{subsection}.}}{1em}{}
% % % % % % % % %

\lstset{language=Java, basicstyle=\footnotesize, numbers=left, frame=tb, breaklines=true, basicstyle=\footnotesize}
\definecolor{pblue}{rgb}{0.13,0.13,1}
\definecolor{pgreen}{rgb}{0,0.5,0}
\definecolor{pred}{rgb}{0.9,0,0}
\definecolor{pgrey}{rgb}{0.46,0.45,0.48}


\newcommand{\ew}[1]		  {\emph{#1}}		% palavras em ingles 
\newcommand{\ns}[1]   	     	  {\mbox{#1}}	         	% comando para não separar palavras
\newcommand{\sigla}[1] 		  {\ns{#1}}			% comando para siglas
\newcommand{\italico}[1] 		  {\textit{#1}}		% itálico
\newcommand{\negrito}[1]  		  {\textbf{#1}}		% negrito
\newcommand{\subl}[1]	   	  {\underline{#1}}		% sublinhado
\newcommand{\X}{\textbullet}			 		% bolinha
\newcommand{\Y}{$\circ$}			 		% não preenchida


\hyphenation{pro-ces-sa-men-to}
\hyphenation{a-pre-sen-ta-da}
\hyphenation{pro-gra-ma}

