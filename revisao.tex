\chapter{Revisão da Literatura}
\citeonline{Moh06} desenvolveram um trabalho utilizando MD no qual o objetivo foi desenvolver modelos para classificação de doenças do arroz egípcio. Um dos algoritmos de aprendizagem utilizado foi a RNA. A RNA foi construída e treinada utilizando uma configuração de 52 entradas, 33 neurônios na camada oculta, 5 saídas, taxa de aprendizagem de 0.3, momento de 0.2 e 500 iterações. O modelo obtido para a previsão de doenças de arroz atingiu um índice de acerto de 96,4\% para o conjunto de dados de teste. Este resultado demonstra a grande eficiência da aplicação de RNAs.

\citeonline{Bla99} realizaram a comparação entre RNA e analise discriminante para criação de classificadores para tipos de coberturas florestais a partir de variáveis cartográficas. O RNA construída utilizou as configurações de 54 entradas, 120 neurônios na camada oculta, 7 classes de tipos de coberturas florestais, com uma taxa de aprendizagem de 0.05, taxa de momento de 0.5 e 1000 interações. Para obter estas configurações para a RNA foram realizados 56 analises diferentes demandando de cerca de 56 horas para cada analise. Após a comparação das técnicas, as RNAs obtiveram uma maior precisão, chegando a 70,58\%.

\citeonline{Ala05} observou que quando o processo de aprendizado de um modelo por meio de algoritmos de aprendizagem é iniciado, este com um grande conjunto de dados, ou com um alto número de repetições, demanda de um alto custo computacional e tempo de execução. Por meio destas observações, Guimarães desenvolveu um aplicativo para distribuir o processamento da construção de seu modelo, por meio do qual o processamento poderia ser realizado por vários computadores, utilizando o algoritmo de aprendizagem Algoritmos Genéticos (AG). Este aplicativo obteve bons resultados conseguindo reduzir seu tempo de execução estimado para construção do modelo de 1450 horas para 84 horas.

\citeonline{Sou11} aplicaram uma ferramenta de MD em paralelo para construção de um modelo de classificação utilizando RNA para produção de soja, afim de observar a relação existente entre os atributos químicos do solo e a produção. Utilizando apenas um computador eles reduziram o tempo de processamento de 280 para 80 segundos. Esses resultados demostraram que a utilização de técnicas de computação paralela podem melhorar significativamente o tempo de resposta das atividades de mineração.
\lstset{language=Java,
  showspaces=false,
  showtabs=false,
  breaklines=true,
  showstringspaces=false,
  breakatwhitespace=true,
  commentstyle=\color{pgreen},
  keywordstyle=\color{pblue},
  stringstyle=\color{pred},
  basicstyle=\ttfamily,
  moredelim=[il][\textcolor{pgrey}]{$$},
  moredelim=[is][\textcolor{pgrey}]{\%\%}{\%\%}
}
\clearpage

\begin{lstlisting}[caption= Exemplo de Código em Java, label=src:java]
/**
 * This is a doc comment.
 */
package com.ociweb.jnb.lombok;

import java.util.Date;
import lombok.Data;
import lombok.EqualsAndHashCode;
import lombok.NonNull;

$$@Data
$$@EqualsAndHashCode(exclude={"address","city","state","zip"})
public class Person {
    enum Gender { Male, Female }

    // another comment

    %%@NonNull%% private String firstName;
    %%@NonNull%% private String lastName;
    %%@NonNull%% private final Gender gender;
    %%@NonNull%% private final Date dateOfBirth;

    private String ssn;
    private String address;
    private String city;
    private String state;
    private String zip;
}
\end{lstlisting}
