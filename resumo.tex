\begin{resumo}

{\noindent

O objetivo deste trabalho é investigar a utilização da computação paralela para reduzir o tempo de resposta da mineração de dados na agricultura. Para esse fim, uma ferramenta, chamada de \textit{Fast Weka} foi definida e implementada. Essa ferramenta permite executar algoritmos de mineração de dados e explorar o paralelismo em computadores multi-núcleos com o uso de \textit{threads} e em grades computacionais empregando redes \textit{peer-to-peer}. A exploração do paralelismo ocorre por meio do paralelismo de dados inerente  ao processo de validação cruzada (\textit{folds}). A ferramenta foi avaliada por meio de experimentos de mineração de dados utilizando algoritmos de redes neurais artificiais aplicados em um conjunto de dados de tipos de coberturas florestais. A computação multi-\textit{thread} e a computação em redes \textit{peer-to-peer} permitem reduzir o tempo de resposta das atividades de mineração de dados. Os melhores resultados são obtidos quando empregados um número múltiplo de \textit{threads} ou pares em relação ao número de \textit{folds} da validação cruzada. Observou-se uma eficiência de 87\% quando utilizadas 4 \textit{threads} para 24 \textit{folds} e 86\% de eficiência também com 24 \textit{folds} utilizando redes \textit{peer-to-peer} com 11 pares. \\

}
{\noindent \textbf{Palavras-chave:} Computação Paralela, Mineração de Dados, \textit{Peer-to-Peer}, Tipos de Coberturas Florestais}

\end{resumo}
